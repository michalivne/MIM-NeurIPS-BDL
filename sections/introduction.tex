<<<<<<< HEAD

\section{Introduction} \label{sec:introduction}

Mutual information is a natural indicator of the quality of a learned representation
\citep{hjelm2018learning}, along with other characteristics, such as the compositionality of
latent factors, that are expected to be useful in downstream tasks, like transfer 
learning \citep{DBLP:journals/corr/BengioTPPB17}.
Mutual information is, however, computationally difficult to estimate for continuous
high-dimensional random variables.  As such, it can be hard to optimize when learning
latent variable models \citep{Hjelm2018,Chen2016}.

% Many existing models are trained using approximate maximum likelihood,
% the canonical example being the variational auto-encoder, or VAE \citep{Kingma2013,Rezende2014}.
% The VAE comprises a probabilistic decoder and an approximate encoder, learned via
% optimization of an evidence-based lower bound (ELBO) on the log marginal data distribution.
% While often producing excellent representations, it is also well-known that the VAE
% sometimes produces pathological results in which the encoder, or approximate
% posterior, conveys relatively little useful information about the latent state.
% This behavior, often referred to as {\em posterior collapse}, results in low
% mutual information between observations and inferred latent states \citep{DBLP:journals/corr/BowmanVVDJB15,ChenKSDDSSA16,
% DBLP:journals/corr/abs-1901-03416,DBLP:journals/corr/OordKK16,
% DBLP:journals/corr/abs-1711-00937}.

This paper formulates a new class of probabilistic auto-encoder model
that is motivated by two key design principles, namely, the maximization
of mutual information, and the symmetry of the encoder-decoder components.
Symmetry captures our desire for both the encoder and decoder to effectively
and consistently model the underlying observation and latent domains.
This is particularly useful for downstream tasks in which either one
or both of the encoder and decoder play a central role.
These properties are formulated in terms of the symmetric Jensen-Shannon
Divergence between the encoder and decoder, combined with an objective term
to maximize mutual information.  We refer to the resulting model as the
{\em mutual information machine}, or MIM.

We contrast MIM with models trained using (approximate) maximum likelihood, 
the canonical example being the variational auto-encoder, or VAE 
\citep{Kingma2013,Rezende2014}. The VAE comprises a probabilistic decoder 
and an approximate encoder, learned via optimization of an evidence-based 
lower bound (ELBO) on the log marginal data distribution.  In contrast to
MIM it is asymmetric in its formulation, and while often producing excellent 
representations, VAEs sometimes produce pathological results in which the 
encoder, or approximate posterior, conveys relatively little information about
the latent state. This behavior, often referred to as {\em posterior collapse}, 
results in low mutual information between observations and inferred latent 
states \citep{DBLP:journals/corr/BowmanVVDJB15,ChenKSDDSSA16,
DBLP:journals/corr/abs-1901-03416,DBLP:journals/corr/OordKK16,
DBLP:journals/corr/abs-1711-00937}.

In what follows we formulate the MIM model and propose a learning
algorithm that minimizes an upper bound on the desired loss.
We show that the resulting objective can be viewed as a symmetrized
form of KL divergence, thereby closely related to the asymmetric KL
objective of the VAE.
In addition to the theoretical development we include several
experiments that reveal properties of MIM through ablation studies,
demonstrating the effectiveness of component terms in the objective.
\micha{review me}
This also enables direct comparisons to the VAE in terms of posterior
collapse, mutual information, data log likelihood, and clustering.
We demonstrate that MIM offers favourable mutual information, similar reconstruction, and better clustering in the latent representation, on the expense of sampling quality, when compared to VAE
with the same architecture. We than demonstrate that given powerful enough model MIM can match the sampling quality of VAE.

%%%%%%%%%%%%%%%%%%%%%%%%%% very old %%%%%%%%%%%%%%%%%%%%%%%%%%%%%%%%%%%%%%%
% Our goal is to formulate a probabilistic encoder-decoder that provides significantly higher
% mutual information than the VAE, and does not suffer from posterior collapse.
% We start by showing in Sec. \ref{sec:posterior-collapse} how maximizing the evidence-lower-bound (ELBO) on maximum likelihood over the
% data marginal $\pjoint(\x)$ can be re-expressed as a form of regularized maximum likelihood over
% a joint density $p(\x,\z)$ with a regularizer that tends to decrease mutual information
% between observations $\x$ and latent states $\z$.


% Finally, the formulation below exhibits an interesting connection between
% the use of  mutual information (MI) and ML in representation learning.
% The use of MI is prevalent in learning latent representations
% \cite{Hjelm2018,Chen2016,Dupont2018,Klys2018}, as it provides a measure of
% the dependence between random variables. Unfortunately, MI is hard to compute;
% it is typically approximated or estimated with non-parametric approaches.
% A detailed analysis is presented in \cite{Hjelm2018},
% which offers scalability with data dimensionality and sample size.
% While it is intuitive and easy to justify the use of MI to enforce a relationship
% between random variables  (\eg, dependency \cite{Chen2016} or independence \cite{Klys2018}),
% MI is often used as a regularizer to extend an existing model.
% The \TZK formulation offers another perspective, showing how MI arises naturally
% with the ML objective, following the assumption that a target distribution can be
% factored into (equally plausible) encoder and decoder models.  We exploit a lower bound
% that allows indirect optimization of MI, without estimating MI directly.
=======
\section{Introduction}
\label{sec:introduction}

The variational auto-encoder (VAE) is a successful class of hierarchical Bayesian models based on transforming data from a simple latent prior into a complicated observation distribution using a neural network.

\begin{align*}
    P(\x) &= \int_\x \Pdec \dz
\end{align*}

This likelihood is in general intractable to compute, there VAEs maximize the so-called evidence lower bound (ELBO)

% Mutual information is a natural indicator of the quality of a learned representation
% \cite{hjelm2018learning}, along with other characteristics, such as the compositionality of
% latent factors, that are expected to be useful in downstream tasks, like transfer 
% learning \cite{DBLP:journals/corr/BengioTPPB17}.
% Mutual information is, however, computationally difficult to estimate for continuous
% high-dimensional random variables.  As such, it can be hard to optimize when learning
% latent variable models \cite{Hjelm2018,Chen2016}.

% % Many existing models are trained using approximate maximum likelihood,
% % the canonical example being the variational auto-encoder, or VAE \cite{Kingma2013,Rezende2014}.
% % The VAE comprises a probabilistic decoder and an approximate encoder, learned via
% % optimization of an evidence-based lower bound (ELBO) on the log marginal data distribution.
% % While often producing excellent representations, it is also well-known that the VAE
% % sometimes produces pathological results in which the encoder, or approximate
% % posterior, conveys relatively little useful information about the latent state.
% % This behavior, often referred to as {\em posterior collapse}, results in low
% % mutual information between observations and inferred latent states \cite{DBLP:journals/corr/BowmanVVDJB15,ChenKSDDSSA16,
% % DBLP:journals/corr/abs-1901-03416,DBLP:journals/corr/OordKK16,
% % DBLP:journals/corr/abs-1711-00937}.

% This paper formulates a new class of probabilistic auto-encoder model
% that is motivated by two key design principles, namely, the maximization
% of mutual information, and the symmetry of the encoder-decoder components.
% Symmetry captures our desire for both the encoder and decoder to effectively
% and consistently model the underlying observation and latent domains.
% This is particularly useful for downstream tasks in which either one
% or both of the encoder and decoder play a central role.
% These properties are formulated in terms of the symmetric Jensen-Shannon
% Divergence between the encoder and decoder, combined with an objective term
% to maximize mutual information.  We refer to the resulting model as the
% {\em mutual information machine}, or MIM.

% We contrast MIM with models trained using (approximate) maximum likelihood, 
% the canonical example being the variational auto-encoder, or VAE 
% \cite{Kingma2013,Rezende2014}. The VAE comprises a probabilistic decoder 
% and an approximate encoder, learned via optimization of an evidence-based 
% lower bound (ELBO) on the log marginal data distribution.  In contrast to
% MIM it is asymmetric in its formulation, and while often producing excellent 
% representations, VAEs sometimes produce pathological results in which the 
% encoder, or approximate posterior, conveys relatively little information about
% the latent state. This behavior, often referred to as {\em posterior collapse}, 
% results in low mutual information between observations and inferred latent 
% states \cite{DBLP:journals/corr/BowmanVVDJB15,ChenKSDDSSA16,
% DBLP:journals/corr/abs-1901-03416,DBLP:journals/corr/OordKK16,
% DBLP:journals/corr/abs-1711-00937}.

% In what follows we formulate the MIM model and propose a learning
% algorithm that minimizes an upper bound on the desired loss.
% We show that the resulting objective can be viewed as a symmetrized
% form of KL divergence, thereby closely related to the asymmetric KL
% objective of the VAE.
% In addition to the theoretical development we include several
% experiments that reveal properties of MIM through ablation studies,
% demonstrating the effectiveness of component terms in the objective.
% \micha{review me}
% This also enables direct comparisons to the VAE in terms of posterior
% collapse, mutual information, data log likelihood, and clustering.
% We demonstrate that MIM offers favourable mutual information, similar reconstruction, and better clustering in the latent representation, on the expense of sampling quality, when compared to VAE
% with the same architecture. We than demonstrate that given powerful enough model MIM can match the sampling quality of VAE.
>>>>>>> overleaf-2019-09-09-1959
